\documentclass[12pt]{article}

\usepackage{sbc-template}
\usepackage[brazil]{babel}
\usepackage{quoting}
\usepackage{setspace}
\usepackage{graphicx,url}
\usepackage{cite}
%\usepackage{float}

\usepackage[brazil]{babel}   
\usepackage[utf8]{inputenc}  
%\usepackage[alf]{abntex2cite}
%\usepackage[style=abnt,backend=biber]{biblatex}

\sloppy

\title{Título Principal}

\author{Dalton Solano dos Reis\inst{1}}

\address{Programa de Pós-Graduação em Computação Aplicada (PPGCAP)\\
Universidade do Estado de Santa Catarina (UDESC)
\email{dalton@furb.br}
}

\begin{document} 
\makeatletter
\renewcommand{\@cite}[2]{({#1\if@tempswa , #2\fi})}
\makeatother


\maketitle

\begin{abstract}
Texto do Abstract.

\textbf{Keywords:} Palavra 1, Palavra 2, Palavra 3.
\end{abstract}
     
\begin{resumo} 
Texto do Resumo.

\textbf{Palavras chaves:} Palavra 1, Palavra 2, Palavra 3.
\end{resumo}

\section{Introdução} \label{sec:introducao}

Texto da Introdução. 

Mais Texto.

\section{Realidade Aumentada} \label{sec:aumentada}

Texto aqui.  

\section{Uso da Realidade Aumentada aplicada à Educação}\label{sec:aplicada}

Texto aqui (Figura \ref{fig:Fig_Cavalli2024}).

\begin{figure}[ht]
\centering
\includegraphics[width=.99\textwidth]{figura.png}
\caption{RA na Química: A) átomos $H_2O$ - B) elemento da água}
\label{fig:Fig_Cavalli2024}
\end{figure}

\section{Considerações Finais}\label{sec:finais}

Texto aqui \cite{Teste2025}.

\bibliographystyle{sbc}
\bibliography{bibliografia}

\end{document}
