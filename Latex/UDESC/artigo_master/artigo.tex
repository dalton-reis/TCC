\documentclass[12pt]{article}

\usepackage{sbc-template}
\usepackage[brazil]{babel}
\usepackage{quoting}
\usepackage{setspace}
\usepackage{graphicx,url}
\usepackage{cite}
%\usepackage{float}

\usepackage[utf8]{inputenc}  
%\usepackage[alf]{abntex2cite}
%\usepackage[style=abnt,backend=biber]{biblatex}

\sloppy

\title{Título Principal}

\author{Dalton Solano dos Reis\inst{1}}

\address{Programa de Pós-Graduação em Computação Aplicada (PPGCAP)\\
Universidade do Estado de Santa Catarina (UDESC)
\email{dalton@furb.br}
}

\begin{document} 
\makeatletter
\renewcommand{\@cite}[2]{({#1\if@tempswa , #2\fi})}
\makeatother


\maketitle

\begin{abstract}
Texto do Abstract.

\textbf{Keywords:} Palavra 1, Palavra 2, Palavra 3.
\end{abstract}
     
\begin{resumo} 
Texto do Resumo.

\textbf{Palavras chaves:} Palavra 1, Palavra 2, Palavra 3.
\end{resumo}

% Dicas
% 	•	\input{arquivo} não insere quebra de página e é ideal para seções pequenas.
% 	•	\include{arquivo} insere quebra de página antes e depois, e gera arquivos .aux individuais (bom para projetos grandes).

% ----------------------------------------------------------
\section{Introdução}\label{sec:introducao}

A introdução apresenta o tema, contextualização, problema e objetivo.  

% Exemplo de citação parentética \cite{albuquerqueToyUserInterfaces2021}.  
Exemplo de citação parentética \parencite{albuquerqueToyUserInterfaces2021}.  

Exemplo de citação narrativa, que segundo
\textcite{aragaoEnsinoProgramacaoPensamento2023}, os jogos sérios podem aumentar
o engajamento.

Exemplo de citação com página
\cite[p.~25]{azumaRecentAdvancesAugmentedReality2001}:
\begin{displayquote}
\small
Com um texto para citação direta que deve ter mais de três linhas de texto
falando sobre alguma coisa qualquer. Assim se tem três linhas de um texto
qualquer, pois se precisa ter todo esse texto.
\end{displayquote}


\section{Realidade Aumentada} \label{sec:aumentada}

Texto aqui.  

\section{Uso da Realidade Aumentada aplicada à Educação}\label{sec:aplicada}

Texto aqui (Figura \ref{fig:Fig_Cavalli2024}).

\begin{figure}[ht]
\centering
\includegraphics[width=.99\textwidth]{figura.png}
\caption{RA na Química: A) átomos $H_2O$ - B) elemento da água}
\label{fig:Fig_Cavalli2024}
\end{figure}

\section{Considerações Finais}\label{sec:finais}

Texto aqui \cite{Teste2025}.

\bibliographystyle{sbc}
\bibliography{bibliografia}

\end{document}
