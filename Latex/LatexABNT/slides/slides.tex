\documentclass[aspectratio=169]{beamer}

% Tema/cores (ajuste ao seu gosto)
\usetheme{Madrid}
\usecolortheme{default}
\setbeamertemplate{navigation symbols}{}
\setbeamertemplate{caption}[numbered]

% ----------------------------------------------------------
% Idiomas e codificação (pdfLaTeX)
\usepackage[english,brazilian]{babel}
\usepackage[utf8]{inputenc}
\usepackage[T1]{fontenc}

% ----------------------------------------------------------
% Gráficos e layout
\usepackage{graphicx,url}
\usepackage{enumitem}
\usepackage{csquotes}
\graphicspath{{./}} % procura figuras no diretório atual

% ----------------------------------------------------------
% Bibliografia ABNT (biber)
\usepackage[style=abnt,backend=biber,language=brazil]{biblatex}
\addbibresource{DR_Estudo.bib}

% ----------------------------------------------------------
\usepackage{listings}
\newcommand{\code}[1]{\texttt{#1}}
\lstset{
  basicstyle=\ttfamily\small,
  columns=fullflexible,
  keepspaces=true,
  upquote=true,
  frame=single,
  breaklines=true
}

% ----------------------------------------------------------
% Título / autoria
\title{Título do Artigo Segundo as Normas da ABNT}
\author{Autor(a) Nome Completo}
\institute{Instituição\\ \href{mailto:autor@exemplo.com}{autor@exemplo.com}}
\date{2025}

% Sumário automático no início de cada seção
% \AtBeginSection[]{
%   \begin{frame}{Roteiro}
%     \tableofcontents[currentsection]
%   \end{frame}
% }

\begin{document}

% ----------------------------------------------------------
\begin{frame}
  \titlepage
\end{frame}

\begin{frame}{Roteiro}
  \tableofcontents
\end{frame}

% ----------------------------------------------------------
\section{Introdução}

\begin{frame}{Introdução}
A introdução apresenta o tema, contextualização, problema e objetivo.

Exemplo de citação parentética: \parencite{albuquerqueToyUserInterfaces2021}.

Exemplo de citação narrativa: \textcite{aragaoEnsinoProgramacaoPensamento2023}
apontam que jogos sérios podem aumentar o engajamento.

Exemplo com página: \parencite[p.~25]{azumaRecentAdvancesAugmentedReality2001}.
\end{frame}

\begin{frame}{Citação direta (trecho longo)}
\small
\begin{displayquote}
Com um texto para citação direta que deve ter mais de três linhas de texto
falando sobre alguma coisa qualquer. Assim se tem três linhas de um texto
qualquer, pois se precisa ter todo esse texto.
\end{displayquote}
\end{frame}

% ----------------------------------------------------------
\section{Conteúdo}

\begin{frame}{Conteúdo}
Texo com o conteúdo.
\end{frame}

% ----------------------------------------------------------
\section{Resultados e Discussão}

\begin{frame}{Resultados (exemplo de figura)}
\begin{figure}
  \includegraphics[width=0.72\linewidth]{figura.png}
  \caption{Exemplo ilustrativo de figura}
\end{figure}
{\tiny Fonte: elaborado pelo autor.}
\end{frame}

% ----------------------------------------------------------
\section{Considerações Finais}

\begin{frame}[fragile]{Considerações Finais}
As conclusões devem relacionar os resultados aos objetivos.

Exemplos:
\begin{itemize}
  \item Classe \texttt{Algoritmo} e método \verb|push_back|.
  \item Use \verb|\texttt| para identificadores e \verb|\verb| para comandos com barras/sublinhados.
\end{itemize}
\end{frame}

% ----------------------------------------------------------
% Referências (estilo ABNT)
\begin{frame}[allowframebreaks]{Referências}
\printbibliography[heading=none]
\end{frame}

\end{document}