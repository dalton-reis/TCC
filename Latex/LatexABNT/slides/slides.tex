% ----------------------------------------------------------
% Modelo de slides Beamer (ABNT via biblatex)
% Compatível com Better BibTeX (.bib) e biber
% ----------------------------------------------------------
\documentclass[aspectratio=169]{beamer}

% Tema/cores (ajuste ao seu gosto)
\usetheme{Madrid}
\usecolortheme{default}
\setbeamertemplate{navigation symbols}{}
\setbeamertemplate{caption}[numbered]

% ----------------------------------------------------------
% Idiomas e codificação (pdfLaTeX)
\usepackage[english,brazilian]{babel}
\usepackage[utf8]{inputenc}
\usepackage[T1]{fontenc}

% ----------------------------------------------------------
% Gráficos e layout
\usepackage{graphicx,url}
\usepackage{enumitem}
\usepackage{csquotes}
\graphicspath{{./}} % procura figuras no diretório atual

% ----------------------------------------------------------
% Bibliografia ABNT (biber)
\usepackage[style=abnt,backend=biber,language=brazil]{biblatex}
\addbibresource{DR_Estudo.bib}

% ----------------------------------------------------------
\usepackage{listings}
\newcommand{\code}[1]{\texttt{#1}}
\lstset{
  basicstyle=\ttfamily\small,
  columns=fullflexible,
  keepspaces=true,
  upquote=true,
  frame=single,
  breaklines=true
}

% ----------------------------------------------------------
% Título / autoria
\title{Título do Artigo Segundo as Normas da ABNT}
\author{Autor(a) Nome Completo}
\institute{Instituição\\ \href{mailto:autor@exemplo.com}{autor@exemplo.com}}
\date{2025}

% Sumário automático no início de cada seção
\AtBeginSection[]{
  \begin{frame}{Roteiro}
    \tableofcontents[currentsection]
  \end{frame}
}

\begin{document}

% ----------------------------------------------------------
\begin{frame}
  \titlepage
\end{frame}

\begin{frame}{Roteiro}
  \tableofcontents
\end{frame}

% ----------------------------------------------------------
% Abstract (EN)
\begin{otherlanguage*}{english}
\begin{frame}{Abstract}
This is the abstract in English. It should succinctly present the article’s
objective, method, results, and conclusions.

\medskip
\textbf{Keywords:} Augmented Reality, Education, Learning.
\end{frame}
\end{otherlanguage*}

% Resumo (PT)
\begin{frame}{Resumo}
Este é o resumo em língua portuguesa. Deve conter uma síntese clara do conteúdo,
incluindo objetivo, método, resultados e conclusões.

\medskip
\textbf{Palavras-chave:} Realidade Aumentada, Educação, Aprendizagem.
\end{frame}

% ----------------------------------------------------------
\section{Introdução}

\begin{frame}{Introdução}
A introdução apresenta o tema, contextualização, problema e objetivo.

Exemplo de citação parentética: \parencite{albuquerqueToyUserInterfaces2021}.

Exemplo de citação narrativa: \textcite{aragaoEnsinoProgramacaoPensamento2023}
apontam que jogos sérios podem aumentar o engajamento.

Exemplo com página: \parencite[p.~25]{azumaRecentAdvancesAugmentedReality2001}.
\end{frame}

\begin{frame}{Citação direta (trecho longo)}
\small
\begin{displayquote}
Com um texto para citação direta que deve ter mais de três linhas de texto
falando sobre alguma coisa qualquer. Assim se tem três linhas de um texto
qualquer, pois se precisa ter todo esse texto.
\end{displayquote}
\end{frame}

% ----------------------------------------------------------
\section{Referencial Teórico}

\begin{frame}{Referencial Teórico}
De acordo com \parencite{pimentelDesignScienceResearch2020}, a pesquisa baseada em
design oferece um processo iterativo de melhoria de artefatos educacionais
(veja Apêndice~\ref{sec:apendiceA}).
\end{frame}

% ----------------------------------------------------------
\section{Metodologia}

\begin{frame}{Metodologia}
Descreve o método, instrumentos e procedimentos adotados:

\begin{enumerate}[label=\alph*), itemsep=2pt]
  \item definição das variáveis;
  \item aplicação dos testes;
  \item análise dos resultados.
\end{enumerate}
\end{frame}

% ----------------------------------------------------------
\section{Resultados e Discussão}

\begin{frame}{Resultados (exemplo de figura)}
\begin{figure}
  \includegraphics[width=0.72\linewidth]{figura.png}
  \caption{Exemplo ilustrativo de figura}
\end{figure}
{\tiny Fonte: elaborado pelo autor.}
\end{frame}

\begin{frame}[fragile]{Resultados (exemplo de código)}
A classe \code{Algorithm} gerencia a lista de \code{logs}.  
A classe \lstinline!Algorithm! valida os blocos.
\begin{lstlisting}[language=Java, caption={Exemplo de classe}, label={lst:ex}]
class Algorithm {
  // ...
}
\end{lstlisting}
\end{frame}

% ----------------------------------------------------------
\section{Considerações Finais}

\begin{frame}{Considerações Finais}
As conclusões devem relacionar os resultados aos objetivos.

Exemplos:
% \begin{itemize}
  % \item Classe \texttt{Algoritmo} e método \verb|push_back|.
%   \item Use \texttt{\textbackslash texttt} para identificadores e \verb|\verb| para comandos com barras/sublinhados.
% \end{itemize}
\end{frame}

% ----------------------------------------------------------
\appendix
\renewcommand{\thesection}{Apêndice \Alph{section}}

\section{Material Suplementar}

\begin{frame}{Exemplo de Apêndice}\label{sec:apendiceA}
Material complementar, questionários ou dados adicionais.
\end{frame}

% ----------------------------------------------------------
% Referências (estilo ABNT)
\begin{frame}[allowframebreaks]{Referências}
\printbibliography[heading=none]
\end{frame}

\end{document}