% ----------------------------------------------------------
% Modelo ABNT para artigo científico (ABNT NBR 6022 / 10520)
% Compatível com arquivos .bib (Better BibLaTeX)
% ----------------------------------------------------------
\documentclass[12pt,oneside]{article}
% \documentclass[12pt,oneside]{abntex2}
% --- Pacotes básicos ---
\usepackage[english,brazil]{babel}
\usepackage[utf8]{inputenc}     % codificação UTF-8
\usepackage[T1]{fontenc}
\usepackage{graphicx,url}
\usepackage{setspace}
\usepackage{enumitem}
% \usepackage{caption}
\usepackage[section]{placeins}  % impede figuras fora da seção

% --- Bibliografia ABNT ---
\usepackage{csquotes}
\usepackage[style=abnt,backend=biber,language=brazil]{biblatex}
% ln -s /Users/daltonreis/GitHub/disciplinas/UDESC_2025/Dr/DR_Estudo.bib
\addbibresource{DR_Estudo.bib}

% --- Ajustes de espaçamento e margens ---
\usepackage[a4paper,margin=2.5cm]{geometry}
% \setstretch{1.5}
\sloppy

\newenvironment{resumoEng}{%
  \par\vspace{1em}%
  \begin{center}\bfseries Abstract\end{center}%
  \begin{quotation}%
}{\end{quotation}}

\newenvironment{resumoPort}{%
  \par\vspace{1em}%
  \begin{center}\bfseries Resumo\end{center}%
  \begin{quotation}%
}{\end{quotation}}

% --- Formatação de título ---
\title{Título do Artigo Segundo as Normas da ABNT}
\author{Autor(a) Nome Completo\thanks{Instituição, e-mail: autor@exemplo.com}}
\date{2025}

\begin{document}
\maketitle

\begin{otherlanguage*}{english}
\begin{resumoEng}
This is the abstract in English. It should succinctly present the article’s
objective, method, results, and conclusions.

\textbf{Keywords:} Augmented Reality, Education, Learning.
\end{resumoEng}
\end{otherlanguage*}

\begin{resumoPort}
Este é o resumo em língua portuguesa. Deve conter uma síntese clara do conteúdo,
incluindo objetivo, método, resultados e conclusões.

\textbf{Palavras-chave:} Realidade Aumentada, Educação, Aprendizagem.
\end{resumoPort}

% ----------------------------------------------------------
\section{Introdução}\label{sec:introducao}

A introdução apresenta o tema, contextualização, problema e objetivo.  

% Exemplo de citação parentética \cite{albuquerqueToyUserInterfaces2021}.  
Exemplo de citação parentética \parencite{albuquerqueToyUserInterfaces2021}.  

Exemplo de citação narrativa, que segundo
\textcite{aragaoEnsinoProgramacaoPensamento2023}, os jogos sérios podem aumentar
o engajamento.

Exemplo de citação com página
\cite[p.~25]{azumaRecentAdvancesAugmentedReality2001}:
\begin{displayquote}
\small
Com um texto para citação direta que deve ter mais de três linhas de texto
falando sobre alguma coisa qualquer. Assim se tem três linhas de um texto
qualquer, pois se precisa ter todo esse texto.
\end{displayquote}


% ----------------------------------------------------------
\section{Referencial Teórico}\label{sec:referencialTeorico}

De acordo com \cite{pimentelDesignScienceResearch2020}, a pesquisa baseada em
design oferece um processo iterativo de melhoria de artefatos educacionais.

% ----------------------------------------------------------
\section{Metodologia}\label{sec:metodologia}
Descreve o método, instrumentos e procedimentos adotados.  
Use listas com \texttt{enumitem}:
\begin{enumerate}[label=\alph*)]
  \item definição das variáveis;
  \item aplicação dos testes;
  \item análise dos resultados.
\end{enumerate}

% ----------------------------------------------------------
\section{Resultados e Discussão}\label{sec:resultadosDiscussao}

Os resultados devem ser apresentados de forma objetiva, com tabelas e figuras
(Figura~\ref{fig:exemplo}).

\begin{figure}[ht]
\centering
\caption{Exemplo ilustrativo de figura}
\fbox{\includegraphics[width=0.7\textwidth]{figura.png}}
\\[2pt]
{\small Fonte: elaborado pelo autor.}
\label{fig:exemplo}
\end{figure}
\vspace{-6pt}

% ----------------------------------------------------------
\section{Considerações Finais}\label{sec:consideracoesFinais}

As conclusões devem relacionar os resultados aos objetivos.  

% ----------------------------------------------------------
\printbibliography

% ----------------------------------------------------------
\appendix
\renewcommand{\thesection}{Apêndice \Alph{section}}

\section{Exemplo de Apêndice}
Material complementar, questionários ou dados adicionais.

\end{document}